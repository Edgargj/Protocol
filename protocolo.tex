%%%%%%%%%%%%%%%%%%%%%%%%%%%%%%%%%%%%%%%%%
% Protocolo Laboratorio de Fisicoquímica Teórica 
% LaTeX Template
% Version 0.1 (10/12/17)
%
% Part of this template was downloaded from:
% http://www.LaTeXTemplates.com
%
% Modifications by:
% Isaias Morales Salazar (Shiver) isaiasms117@gmail.com
% Luis Alfredo Nuñez (Fdx) luis.alfredo.nu@gmail.com
% Edgar García Juarez (b613) edgargj.16@gmail.com
%%%%%%%%%%%%%%%%%%%%%%%%%%%%%%%%%%%%%%%%%
%-----------------------------------------------------------------------------% 
% Packages
%-----------------------------------------------------------------------------%
\documentclass[12pt]{article} 	
\usepackage[utf8]{inputenc}
\usepackage[spanish]{babel}
\usepackage[usenames]{color}
\usepackage{float}
\usepackage{graphicx}
\usepackage{amsmath,mathtools}
\usepackage[font={small},format=plain,labelfont=bf]{caption} % Size and format of foot of figures and tables
\usepackage{color}
\usepackage{hyperref}
%-----------------------------------------------------------------------------%
% Margin settings 
%-----------------------------------------------------------------------------%
\usepackage{geometry}
\geometry{
	paper=letterpaper, % Paper type
	inner=2.5cm, % Inner margin
	outer=2.5cm, % Outer margin
	top=3.5cm, % Top margin
	bottom=3.5cm, % Bottom margin
}
%-----------------------------------------------------------------------------% 
% New Commands
%-----------------------------------------------------------------------------%
\renewcommand{\tablename}{Tabla} % Name of table foot 
\renewcommand{\baselinestretch}{1.5} % space between lines 1.5
\newcommand\tab[1][0.5cm]{\hspace*{#1}}
\newcommand\vtab[1][0.5cm]{\vspace*{#1}}
%-----------------------------------------------------------------------------% 
%-----------------------------------------------------------------------------%
% Document
%-----------------------------------------------------------------------------%
%-----------------------------------------------------------------------------% 

\begin{document}
%-----------------------------------------------------------------------------%
% Cover Page
%-----------------------------------------------------------------------------% 
\pagestyle{empty} 
\begin{center}
\includegraphics[width=4cm]{Images/buap2014.png}

BENEMÉRITA UNIVERSIDAD AUTÓNOMA DE PUEBLA\\
\rule{150mm}{0.1mm}
\begin{small}
FACULTAD DE CIENCIAS QUÍMICAS\\
LABORATORIO DE MODELADO MATEMÁTICO\\
Y DESAROLLO DE SOFTWARE CIENTÍFICO\\
Y LABORATORIO DE FISCOQUÍMICA ORGÁNICA TEÓRICA\\
\end{small}
\rule{150mm}{0.1mm}

\vtab[.1cm]
\Large{Protocolo de Tesis}\\
\vtab[0.2cm]

\large{\textbf{Título de tesis}}\\
\vtab[0.2cm]

PRESENTA \\
Édgar García Juárez \\
\vtab[0.6cm]
\begin{tabular}{cc}
DIRECTOR DE TESIS & CODIRECTOR DE TESIS \\ 
Dr. Juan Manuel Solano Altamirano & Dr. Julio Manuel Hernández Pérez \\ 
FCQ-BUAP & FCQ-BUAP \\ 
\end{tabular} 
\vtab[1cm] \\
\large{Coordinador del Departamento de Fisicoquímica\\
Dr. Juan Carlos Ramírez García}\\
\vtab[0.4cm]
Marzo de 2022 \\
\end{center}

\newpage

%-----------------------------------------------------------------------------% 
% Introducción
%-----------------------------------------------------------------------------% 
\pagestyle{plain} 

\section*{Introducción}
Los ordenadores son máquinas potentes que podemos utilizar para resolver problemas de la vida
cotidiana. Sin embargo, su propósito original fue la implementación de funciones aritméticas
en los circuitos electrónicos que contiene para, posteriormente, configurarse y hacer cálculos
algebráicos en ella. Este fue uno de los grandes logros de la comunidad científica en el siglo XX.
Actualmente, el uso de los ordenadores en las ciencias es fundamental, al grado de que el desarrollo
del cómputo creó una nueva rama de la química, la \textbf{química computacional}.
Adicionalmente, la sorprendente evolución de las matemáticas, la física y la química teórica han contribuido al florecimiento de la química computacional, al proveernos de conceptos, modelos
teóricos, métodos numéricos y analíticos más eficientes que se han incorporado en algoritmos
programables. Así,en nuestros días, es posible calcular geometrías moleculares, equilibrios de
reacciones, espectros y otras propiedades físicas y químicas con las herramientas de esta nueva
rama. 

En la naturaleza y en el entorno científico, existen compuestos tan reactivos que no pueden
aislarse, por lo que no pueden estudiarse mediante técnicas comunes de laboratorio. Sin embargo, este tipo de moléculas sí pueden estudiarse con métodos computacionales. Desde luego, no debe considerase a la química computacional como una rival de las técnicas experimentales tradicionales, sino como
una aliada, ya que cuando se utilizan ambas, se logran resultados que serían imposibles de obtener si se utilizasen de forma excluyente. Podemos decir, entonces, que la química computacional es una disciplina que comprende los aspectos de la investigación en química que se benefician de la aplicación de los ordenadores \cite{Cuevas2003}.

Las simulaciones efectuadas por computadoras tienen varias ventajas, entre éstas tenemos:
\begin{enumerate}
\item Son más económicas y rápidas que los experimentos físicos.
\item Pueden resolver un amplio margen de problemas, comparado con los que se podrían resolver con equipos de laboratorio específicos o de tecnología actual. 
\item Los cálculos están limitados sólo por la velocidad del ordenador y la capacidad de memoria de éste.
\end{enumerate}

Para los fines de este proyecto, la química computacional puede dividirse en dos categorías, los métodos basados en la mecánica molecular y los métodos basados en la mecánica cuántica.
Los métodos basados en la mecánica molecular (MM) se fundamentan en las leyes de la mecánica
clásica, y consideran a los átomos como partículas puntuales dotadas de masa y carga, unidos
entre sí por enlaces que pueden modelarse como resortes.

Por otro lado, los métodos basados en la mecánica cuántica, resuelven la ecuación de Schrödinger
y utilizan la función de onda resultante
para describir a los sistemas.
En particular, con estos métodos se modela a las moléculas mediante un tratamiento directo
de la estructura electrónica. Dicha estructura se puede abordar a través de métodos
\textit{ab initio} (significa ``desde el principio'' y se refiere a que en
este tipo de cálculos se emplean constantes fundamentales de la física),
o bien a través de métodos  \textit{semiempíricos} (emplea parámetros cuyos valores se ajustan con
datos experimentales de cálculos \textit{ab initio}). Tanto los métodos \textit{ab initio} como los
métodos \textit{semiempíricos} se enfocan en predecir las propiedades de los sistemas atómicos y
moleculares. En este sentido, no existe un método intrínsecamente ``mejor'' que otro, y en cambio,
todo dependerá del sistema a analizar, los recursos de cómputo disponibles y la precisión requerida. Adicionalmente, existen dos factores importantes para elegir un método de cálculo adecuado: la naturaleza de la molécula y los parámetros conocidos de la molécula \cite{Cuevas2003}. 

Dentro de las magnitudes más relevantes que pueden determinarse, con cierta facilidad, empleando
cálculos de estructura electrónica se encuentran la entalpía de formación, la entropía
y la energía libre de Gibbs. Estas propiedades se consideran relevantes porque brindan información acerca de la estabilidad y la termodinámica de las moléculas, y porque a partir de ellas, es posible entender varios fenómenos que ocurren en los procesos químicos.

Se denomina \textbf{entalpía de formación} (representada como $\Delta H_{f}^{\circ}$), a la energía
involucrada en la reacción química que relaciona la formación de 1 mol de un compuesto a partir de sus elementos en su forma más estable.
Una forma experimental común de determinar la $\Delta H_{f}^{\circ}$, consiste en calcinar un compuesto dentro de una bomba calorimétrica y cuantificar el cambio de temperatura,
con el fin de medir la cantidad de calor involucrado en esa reacción. Teóricamente, es posible obtener la entalpía de formación haciendo uso de tablas. (Existen extensas tabulaciones de entalpías de formación determinadas experimentalmente  \cite{NIST1998, Tajti2004}.) No obstante, también es posible cuantificar la entalpía de formación mediante cálculos \textit{ab initio}  \cite{Lewars2016}.
Esta última opción es valiosa porque (1) es mucho más sencillo y económico que hacer un experimento termoquímico, (2) existen compuestos que no han sido medidos ni tabulados y (3) hay compuestos que son altamente reactivos, o compuestos de interés biológico que están disponibles sólo en pequeñas cantidades, por lo que no es posible someterlos a rígidos protocolos experimentales, \textit{v.gr.} reacciones de combustión  \cite{Lewars2016}.

\section*{Antecedentes}
La precisión de un cálculo computacional, en particular en la energía, varía notablemente con el nivel de teoría y con el tipo de base utilizados para realizar el cálculo.
Afortunadamente, existen metodologías que permiten conocer la energía con una precisión de
hasta $\pm$ 2kJ/mol, respecto a una determinación experimental. Estos métodos constan de secuencias
de cálculos predefinidos y fueron desarrollados específicamente para lograr valores muy precisos
con costos computacionales aceptables (véase p. ej. los métodos introducidos por Pople y otros
autores  \cite{Cuevas2003}). Una categoría muy popular de estos métodos combinados es la que está
conformada por las denominadas teorías gaussian-n.
Éstas se usan para calcular energías en sistemas moleculares que contienen átomos desde
el hidrógeno hasta el cloro, y su objetivo es desarrollar procedimientos generales,
de amplia aplicabilidad para cualquier molécula, y ser capaces de reproducir valores
termoquímicos experimentales con la precisión mencionada arriba.
Algunos de esos métodos son: Gn (G1, G2, G3, G4).  Existen otras técnicas como CBS-N
(CBS-APNO y CBS-QB3)  \cite{Simmie2015}, pero en este trabajo abordaremos exclusivamente
los métodos G3 y G4 del software \textbf{Gaussian}.

El método G4 es un procedimiento en el que se realiza una secuencia de cálculos sustentados
en la teoría de orbitales moleculares  (\textit{ab initio}) y ha sido empleada frecuentemente
en el cálculo de energías de enlace, entalpías de formación, potenciales de ionización
y afinidades electrónicas  \cite{Cuevas2003, Tajti2004}.

Los pasos realizados en un cálculo con el método G4 son los siguientes:
\begin{enumerate}
        \item Obtención de la estructura de equilibrio a nivel B3LYP/6-31G(2\textit{df,p}).
        \item Especificación de las frecuencias armónicas también a nivel B3LYP/6-31G(2\textit{df,p}).
        \item Determinación de límite de energía Hartree-Fock.
        \item Corrección de la energía.
        \item Evaluación de la energía MP4/6-31G(\textit{d}), correcciones del paso anterior y combinación aditiva con el paso 3.
        \item Especificación de un nivel alto (HLC) con parámetros empíricos.
        \item Obtención de la energía total a T = 0 K con una corrección en la energía de punto zero obtenida en el paso 2.
\end{enumerate}

Estos pasos se utilizan para ensamblar la energía molecular resultante como una suma de varias diferencias de energía y un incremento empírico en la energía final (la ``corrección de nivel superior'') basado en el número de electrones apareados y no apareados  \cite{Curtiss2007}. Este proceso proporciona la energía de la molécula a 0 K.

Una vez conocida la energía molecular, se puede calcular la entalpía de formación de la molécula
en cuestión a 0 K. Para esto, se puede utilizar alguno de los tres enfoques más comúnes:
el método de atomización, el método de formación o  el método de reacción isodésmica.
De los tres enfoques anteriores, el método de atomización da mejores resultados, especialmente
para moléculas orgánicas, y es conceptualmente el más sencillo, ya que consiste en romper
los enlaces de la molécula para obtener a sus atomos en fase gasesosa  \cite{Nicolaides1996}. Ahora bien, es común que las entalpías de formación estén tabuladas en condiciones normales de
temperatura y presión, por lo que después de obtener las energías a 0 K, es necesario calcularlas
a 298.15 K. Esto se puede hacer mediante la Termodinámica Estadística.

En efecto, la Termodinámica Estadística permite obtener el valor de la energía interna
de una molécula, a 298.15 K, mediante la función de partición de un gas poliatómico. Dicha
función resulta ser un producto de funciones de partición relacionadas con movimientos
traslacionales, rotacionales y vibracionales  \cite{McQuarrie1976, Nicolaides1996}.
Brevemente, el término translacional aporta $\frac{3}{2}$ RT, el término
rotacional aporta $\frac{3}{2}$ RT (aunque sólo RT si la molécula es lineal) y el rotacional
contribuye con RT. Estas cantidades son resultado de considerar a las moléculas como si
fueran partículas libres en una caja para el movimiento translacional, la aproximación del rotor
rigido para el movimiento rotacional y el oscilador armónico para el movimiento vibracional.
Adicionalmente, pueden incorporarse correcciones más finas como la aproximación de Nicolaides y otros autores  \cite{McQuarrie1976, Nicolaides1996}.

El procedimiento descrito anteriormente suele ser bastante tedioso si se realiza manualmente,
por lo que resuta indispensable contar con una herramienta que permita realizar
las correcciones térmicas de manera automática. Dicha herramienta se materializa en un
algoritmo de cómputo que determine la entalpía de formación de compuestos orgánicos
a T = 298.15 K, utilizando archivos de salida del software \textit{Gaussian}, y de manera que
se añadan a voluntad correcciones finas como las de Nicolaides.
 
%-----------------------------------------------------------------------------% 
% Objetivos 
%-----------------------------------------------------------------------------% 
\section*{Objetivos}

\subsection*{Objetivo General}

Crear un conjunto de programas de cómputo científico que determinen la entalpía de formación de compuestos orgánicos a T = 298.15 K con correciones en la energía interna, empleando archivos de salida del software científico \textit{Gaussian}.

\subsection*{Objetivos Particulares}
\begin{itemize}
\item Diseñar e implementar programas de cómputo que se puedan utilizar a través de la línea de comandos en un sistema operativo de GNU/Linux.

\item Implementar dichos programas con un lenguaje de programación orientado a objetos, de
manera que sea posible modularizar el código en partes independientes.
\end{itemize}

\section*{Justificación}
El desarollo de software científico de alto rendimiento en Termoquímica es una de
las principales líneas de investigación de nuestro grupo. Por lo tanto, el cálculo
de funciones termodinámicas de forma optimizada, dará como resultado una alta
eficiencia en el flujo de trabajo, además de que
reducirá considerablemente la ocurrencia de errores humanos. 


%-----------------------------------------------------------------------------% 
% Metodología
%-----------------------------------------------------------------------------% 
\section*{Metodología}
Se utilizará el lenguaje de programación C++ para implementar los programas descritos anteriormente. El motivo principal de esta elección es que C++ es un lenguage altamente compatible con el
paradigma de la programación orientada a objetos. Una característica importante de este paradigma
es que permite recicla y reutilizar el código en proyectos futuros \cite{cplusplus}. A continuación, se explican las diferentes clases que existirán en los programas desarrollados en este proyecto.

\subsection*{Clases}
Ahora, se enlistan los nombres de las clases para estos programas:
\begin{itemize}
	\item Enthalpyinputdata. Se encargará de leer los datos (provenientes de Gaussian) del archivo de entrada tales como tipo de método, número de especies atómicas, energía G4 a 0 K, Entalpía G4 a 0 K,
	número atómico y número de átomos, tipo de molécula (lineal o no lineal), tipo de
	aproximación usada (p.ej. Nicolaides o Rotor Rígido y Oscilador Armónico), número de
	modos de vibración y frecuencias vibracionales de la molécula. La finalidad de esta clase
	será determinar la información indispensable para comenzar con el cálculo de la
	entalpía de formación.
	\item Method. La clase Method se utilizará para seleccionar el tipo de método que
	se realizó en \textit{Gaussian}, y así, devolver valores específicos
	para los átomos existentes en la molécula analizada.
	\item EnthalpyG4 y variantes EnthalpyG3, EnthalpyG3MP2, EnthalpyCBS-APNO y
	\-Ent\-halp\-y\-CBSt-QB3.
	El trabajo de la clase EnthalpyG4 y cada una de sus variantes será realizar las operaciones 
	aritméticas necesarias para obtener el calor de formación de la  molécula,
	la entalpía de formación a
	$T$ = 298.15 K (a través del método de atomización) y la entalpía de formación a
	$T$ = 298.15 K con correcciones a la energía interna.
	Además, esta clase imprimirá los resultados a través de la línea de comandos.
\end{itemize}

\newpage


%-----------------------------------------------------------------------------% 
% Lugares y Cronograma
%-----------------------------------------------------------------------------% 
\section*{Lugar y tiempo de realización}
El trabajo de tesis se llevará a cabo en el Laboratorio de Modelado Matemático y Desarollo de Software Científico y en el Laboratorio de Fisicoquímica Orgánica Teórica de la Facultad de Ciencias Químicas de la BUAP, en un periodo de 12 meses, de acuerdo con el siguiente conograma.
\vspace{1cm}
\begin{table}[hbp!]
\centering
\footnotesize
\setlength{\tabcolsep}{2.0pt}
\begin{tabular}{||p{0.4\linewidth}|c|c|c|c|c|c|c|c|c|c|c|c||}
\hline
\textbf{Actividad} & \multicolumn{12}{|c||}{2021-2022}\\\hline
& Ago & Sep & Oct & Nov & Dic & Ene & Feb & Mar & Abr & May & Jun & Jul\\\hline
Revisión bibliográfica & X & X & X & X & X & X & X & X & X & X & X &  \\\hline
Curso de Termodinámica Estadística & X & X & X & X & & & & & & & &  \\\hline
Cursos de \texttt{Bash y C++} & X & X & X & X & & & & & & & &  \\\hline
Inspección de código inicial & & & X & X & X & X & & & & & & \\\hline
Redacción de tesis &  &  &  &  & X & X & X & X & & & & \\\hline
Periodo de pruebas y correción de bugs & & & & & & & X & X & & & & \\\hline
Verificación y corrección de tesis & & & & & & & & & X & X & X & \\\hline
Examen &  &  &  &  &  &  &  &  &  &  & X  & \\\hline
\end{tabular}
\end{table}
\newpage
%-----------------------------------------------------------------------------% 
% Bibliografía
%-----------------------------------------------------------------------------% 
\bibliographystyle{./general} 
\bibliography{./listado}
\end{document}
