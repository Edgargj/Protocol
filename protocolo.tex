%%%%%%%%%%%%%%%%%%%%%%%%%%%%%%%%%%%%%%%%%
% Protocolo Laboratorio de Fisicoquímica Teórica 
% LaTeX Template
% Version 0.1 (10/12/17)
%
% Part of this template was downloaded from:
% http://www.LaTeXTemplates.com
%
% Modifications by:
% Isaias Morales Salazar (Shiver) isaiasms117@gmail.com
% Luis Alfredo Nuñez (Fdx) luis.alfredo.nu@gmail.com
% Edgar García Juarez (b613) edgargj.16@gmail.com
%%%%%%%%%%%%%%%%%%%%%%%%%%%%%%%%%%%%%%%%%
%-----------------------------------------------------------------------------% 
% Packages
%-----------------------------------------------------------------------------%
\documentclass[12pt]{article} 	
\usepackage[utf8]{inputenc}
\usepackage[spanish]{babel}
\usepackage[usenames]{color}
\usepackage{float}
\usepackage{graphicx}
\usepackage{amsmath,mathtools}
\usepackage[font={small},format=plain,labelfont=bf]{caption} % Size and format of foot of figures and tables
\usepackage{color}
\usepackage{hyperref}
%-----------------------------------------------------------------------------%
% Margin settings 
%-----------------------------------------------------------------------------%
\usepackage{geometry}
\geometry{
	paper=letterpaper, % Paper type
	inner=2.5cm, % Inner margin
	outer=2.5cm, % Outer margin
	top=3.5cm, % Top margin
	bottom=3.5cm, % Bottom margin
}
%-----------------------------------------------------------------------------% 
% New Commands
%-----------------------------------------------------------------------------%
\renewcommand{\tablename}{Tabla} % Name of table foot 
\renewcommand{\baselinestretch}{1.5} % space between lines 1.5
\newcommand\tab[1][0.5cm]{\hspace*{#1}}
\newcommand\vtab[1][0.5cm]{\vspace*{#1}}
%-----------------------------------------------------------------------------% 
%-----------------------------------------------------------------------------%
% Document
%-----------------------------------------------------------------------------%
%-----------------------------------------------------------------------------% 

\begin{document}
%-----------------------------------------------------------------------------%
% Cover Page
%-----------------------------------------------------------------------------% 
\pagestyle{empty} 
\begin{center}
\includegraphics[width=4cm]{Images/buap2014.png}

BENEMÉRITA UNIVERSIDAD AUTÓNOMA DE PUEBLA\\
\rule{150mm}{0.1mm}
\begin{small}
FACULTAD DE CIENCIAS QUÍMICAS\\
LABORATORIO DE MODELADO MATEMÁTICO\\
Y DESAROLLO DE SOFTWARE CIENTÍFICO\\
Y LABORATORIO DE FISCOQUÍMICA ORGÁNICA TEÓRICA\\
\end{small}
\rule{150mm}{0.1mm}

\vtab[.1cm]
\Large{Protocolo de Tesis}\\
\vtab[0.2cm]

\large{\textbf{Título de tesis}}\\
\vtab[0.2cm]

PRESENTA \\
Édgar García Juárez \\
\vtab[0.6cm]
\begin{tabular}{cc}
DIRECTOR DE TESIS & CODIRECTOR DE TESIS \\ 
Dr. Juan Manuel Solano Altamirano & Dr. Julio Manuel Hernández Pérez \\ 
FCQ-BUAP & FCQ-BUAP \\ 
\end{tabular} 
\vtab[1cm] \\
\large{Coordinador del Departamento de Fisicoquímica\\
Dr. Juan Carlos Ramírez García}\\
\vtab[0.4cm]
Marzo de 2022 \\
\end{center}

\newpage

%-----------------------------------------------------------------------------% 
% Introducción
%-----------------------------------------------------------------------------% 
\pagestyle{plain} 

\section*{Introducción}
Los ordenadores son máquinas potentes que podemos utilizar para resolver problemas de la vida cotidiana, sin embargo, su propósito original fue la implementación de funciones aritméticas en los circuitos electrónicos que contiene para, posteriormente, configurarse y hacer cálculos algebráicos en ella. Este fue uno de los grandes logros de la comunidad científica en el siglo XX. Actualmente, su uso en las ciencias es fundamental y el desarrollo del cómputo creó una nueva rama de la química, la \textbf{química computacional}. La sorprendente evolución de las matemáticas, la física y la química teórica han contribuido al florecimiento de la química computacional, al proveernos de conceptos, modelos teóricos, métodos numéricos y analíticos más eficientes que se han incorporado en algoritmos programables. Es posible calcular geometrías moleculares, equilibrios de reacciones, espectros y otras propiedades físicas y químicas con las herramientas de esta nueva rama. Existen compuestos que son muy reactivos para ser aislados y por eso, son incapaces de ser estudiados por técnicas comunes de laboratorio. Este tipo de moléculas pueden estudiarse con métodos computacionales, desde luego, no debe considerase a la química computacional como una rival de las técnicas experimentales tradicionales, ambas logran resultados que son imposibles de obtener de forma individual. Podemos decir entonces, que la química computacional es una disciplina que comprende todos aquellos aspectos de la investigación en química que se benefician de la aplicación de los ordenadores \cite{Cuevas2003}. 
Las simulaciones efectuadas por computadoras tienen varias ventajas:
\begin{enumerate}
\item Son más económicas y rápidas que los experimentos físicos.
\item Pueden resolver un amplio margen de problemas que los que se podrían resolver con equipos de laboratorio específicos o de tecnología actual. 
\item Los cálculos sólo están limitados por la velocidad del ordenador y la capacidad de memoria de éste.
\end{enumerate}\\

La química computacional puede dividirse en dos categorías; métodos de mecánica molecular y métodos de mecánica cuántica. Los métodos de mecánica molecular (MM) se fundamentan en las leyes de la mecánica clásica (consideran a los átomos como partículas puntuales dotadas de masa y carga, unidos por enlaces que pueden ser tratados como resortes). \\

Por otro lado, los métodos que tienen como fundamento la mecánica cuántica, emplean la ecuación de Schrödinger para describir una molécula con un tratamiento directo de la estructura electrónica abordada a través de métodos \textit{ab initio} (significa “desde el principio” y se refiere a que en este tipo de cálculos se emplean constantes fundamentales de la física) y métodos  \textit{semiempíricos} (emplea parámetros cuyos valores se ajustan con datos experimentales de cálculos \textit{ab initio}). Los métodos \textit{ab initio} y los métodos \textit{semiempíricos}, se enfocan en las predicciones de las propiedades de los sistemas atómicos y moleculares. No existe un método "mejor" que otro porque todo dependerá del sistema a analizar, los recursos de cómputo disponibles y la precisión requerida. Adicionalmente, existen dos factores importantes para elegir un método de cálculo adecuado; la naturaleza de la molécula y los parámetros conocidos de la molécula\cite{Cuevas2003}. \\

Las magnitudes más relevantes que pueden ser determinadas con cierta facilidad empleando cálculos de estructura electrónica son:

\begin{enumerate}
\item Entalpía de formación.
\item Entropía.
\item Energía libre de Gibbs.
\end{enumerate}

Estos valores son fundamental porque brindan información acerca de la estabilidad y la termodinámica de las moléculas. A partir de ello, es posible entender los efectos que tienen los procesos químicos. \\ 

Cuándo, en una reacción se obtiene 1 mol de un compuesto a partir de sus elementos en su forma más estable, a la energía involucrada en la formación de calor se le conoce como \textbf{entalpía de formación} y se representa como $\Delta H_{f}^{\circ}$. Una forma de determinar la $\Delta H_{f}^{\circ}$ es calcinar un compuesto dentro de una bomba calorimétrica y cuantificar el cambio de temperatura, para calcular la cantidad de calor involucrado en esa reacción. Teóricamente, es posible obtener la entalpía de formación haciendo uso de tablas, existen extensas tabulaciones de entalpías de formación determinadas experimentalmente\cite{NIST1998, Tajti2004}. No obstante, también es posible cuantificar la entalpía de formación mediante cálculos \textit{ab initio} \cite{Lewars2016}. Esta última opción es valiosa porque (1) es mucho más sencillo y económico que hacer un experimento termoquímico, (2) existen compuestos que no han sido medidos ni tabulados y (3) hay compuestos que son altamente reactivos, o compuestos de interés biológico que están disponibles sólo en pequeñas cantidades, por lo tanto, no es posible someterlos a rígidos protocolos experimentales, \textit{v.gr.} reacciones de combustión \cite{Lewars2016}.

\section*{Antecedentes}
La precisión de un cálculo computacional en la energía, varía notablemente con el nivel de teoría y con el tipo de base de cálculo ocupada. Afortunadamente, existen metodologías que permiten conocer la energía con una precisión de $\pm$ 2kJ/mol, respecto a una determinación experimental. Estos métodos constan de secuencias de cálculos predefinidos y son combinados para lograr valores muy precisos con costos computacionales aceptables, fueron introducidos por Pople y otros autores \cite{Cuevas2003}. Una categoría muy popular son las denominadas teorías gaussian-n y se usan para el cálculo de energías en sistemas moleculares que contienen átomos desde el hidrógeno hasta el de cloro, su objetivo es desarrollar procedimientos generales, de amplia aplicabilidad para cualquier molécula y ser capaz de reproducir valores termoquímicos experimentales con la precisión antes mencionada. Algunos de esos métodos son: Gn (G1, G2, G3, G4). \\

Existen otras técnicas como CBS-N (CBS-APNO y CBS-QB3) \cite{Simmie2015}, pero en este trabajo abordaremos exclusivamente los método Gn del software \textbf{Gaussian}. En términos prácticos hay cuatro pasos generales en estos procedimientos:
\begin{enumerate}
        \item Elección de la geometría molecular a un nivel bajo (Hartree-Fock o B3LYP).
        \item Selección de una base.
        \item Especificación de la energía de correlación electrónica.
        \item Determinación de las constantes rotacionales, traslacionales y vibracionales.
\end{enumerate}
La teoría G4 es un procedimiento en el que se realiza una secuencia de cálculos sustentados en la teoría de orbitales moleculares  (\textit{ab initio}) y ha sido muy empleada en el cálculo de energías de enlace, entalpías de formación, potenciales de ionización y afinidades electrónicas \cite{Cuevas2003, Tajti2004}. Los pasos en un cálculo en la teoría G4 son los siguientes:
\begin{enumerate}
        \item Obtención de la estructura de equilibrio en un nivel B3LYP/6-31G(2\textit{df,p}).
        \item Especificación de las frecuencias armónicas.
        \item Determinación de límite de energía Hartree-Fock.
        \item Corrección de la energía.
        \item Evaluación de la energía MP4/6-31G(\textit{d}), correcciones del paso anterior y combinación aditiva con el paso 3.
        \item Especificación de un nivel alto (HLC) con parámetros empíricos.
        \item Obtención de la energía total a T = 0 K con una corrección en la energía de punto zero obtenida en el paso 2.
\end{enumerate}

Estos siete pasos son utilizados para ensamblar la energía molecular como una suma de varias diferencias de energía y un incremento empírico en la energía final (la ``corrección de nivel superior") basado en el número de electrones apareados y no apareados\cite{Curtiss2007}.

En química computacional hay tres enfoques principales que predicen entalpías de formación utilizando la teoria Gn.

\begin{itemize}
\item Método de atomización.
\item Método de formación.
\item Método de reacción isodésmica.
\end{itemize}

De los 3 enfoques anteriores, el método de atomización da mejores resultados, especialmente para moléculas orgánicas y es concenpualmente el más sencillo, porque se basa en la ruptura de enlaces de una molécula para obtener a sus atomos en fase gasesosa\cite{Nicolaides1996}. Sin embargo, el ajuste de la entalpía de formación de 0 K a 298.15 K, que es la diferencia entre las dos cantidades del método G4 proporcionadas en el resumen termoquímico al final de cálculo, puede ser mejorada al hacer correciones térmicas. La Termodinámica Estadística permite obtener el valor de la  energía interna a 298.15 K al separar la función de partición en un producto de sus componentes traslacionales, rotacionales, vibracionales, electrónicos y nucleares\cite{McQuarrie1976, Nicolaides1996}. En dichas componentes existen aportaciones de $\frac{3}{2}$ RT para la contribución traslacional, $\frac{3}{2}$ RT para la contribución rotacional (RT para moléculas lineales) y un RT adicional para convertir la energía en entalpía (el llamado término PV). Además, la aproximación del rotor rigido - oscilador armónico puede ser utilizada para el componente vibracional junto con la aproximación de Nicolaides y otros autores \cite{McQuarrie1976, Nicolaides1996}. Las contribuciones de los términos electrónicos y nucleares son ignorados (es decir, las funciones de partición correspondientes se establecen en la unidad). No obstante, este procedimiento suelen ser demasiado tediosos cuando se examinan de forma manual (por la exagerada cantidad de expresiones algebráicas que deben ser evaluadas de forma continua). Por lo tanto, es necesario contar con una herramienta que permita optimizar el tiempo que requieren estos cálculos.\\

En relación con todo lo descrito anteriormente, es necesario un algoritmo de cómputo que determine la entalpía de formación de compuestos orgánicos a T = 298.15 K con correciones en la energía interna, utilizando archivos de salida del software \textit{Gaussian}.
 
%-----------------------------------------------------------------------------% 
% Objetivos 
%-----------------------------------------------------------------------------% 
\section*{Objetivos}

\subsection*{Objetivo General}

Crear un conjunto de programas de cómputo científico que determinen la entalpía de formación de compuestos orgánicos a T = 298.15 K con correciones en la energía interna, empleando archivos de salida del software científico \textit{Gaussian}.

\subsection*{Objetivos Particulares}
\begin{itemize}
\item Diseñar una programación que sea capaz de utilizarse a través de una línea de comandos en un sistema operativo de GNU/Linux.

\item Implementar en dicho programas un lenguaje de programación orientado a objetos que permita fragmentar el código en partes independientes para futuros proyectos.
\end{itemize}

\section*{Justificación}
El desarollo de software científico de alto rendimiento en Termoquímica es una de las principales líneas de investigación de nuestro grupo. Por lo tanto, el cálculo de funciones termodinámicas de forma optimizada, dará como resultado una alta eficiencia en el flujo de trabajo. 


%-----------------------------------------------------------------------------% 
% Metodología
%-----------------------------------------------------------------------------% 
\section*{Metodología}
El lenguaje de programación que será utilizado para la creación de estos programas se conoce como c++. El motivo principal de su uso, será la implementación de una programación orientada a objetos que fragmentará el código en partes independientes, permitiendo así, reciclar el código para futuros proyectos \cite{cplusplus}. A continuación, se explican las diferentes clases que existirán en estos programas.

\subsection*{Clases}
Ahora, se enlistan los nombres de las clases para estos programas:
\begin{itemize}
	\item Enthalpyinputdata.
	\item Method.
	\item EnthalpyG4.
	\item EnthalpyG3.
	\item EnthalpyG3MP2.
	\item EnthalpyCBS-APNO.
	\item EnthalpyCBS-QB3.
\end{itemize}

\subsection*{Clase Enthalpyinputdata}
Enthalpyinputdata se encargará de leer los datos (provenientes de Gaussian) del archivo de entrada. Los datos que lee esta clase son:
\begin{itemize}
	\item Tipo de método.
	\item Número de especies atómicas.
	\item Energía G4 a 0 K.
	\item Entalpía G4 a 0 K.
	\item Número atómico y número de átomos.
	\item Tipo de molécula (lineal o no lineal).
	\item Tipo de aproximación usada (Nicolaides y otros autores o Rotor Rígido y Oscilador Armónico).
	\item Número de modos de vibración.
	\item Frecuencias vibracionales de la molécula.
\end{itemize}
La finalidad de esta clase será determinar la información indispensable para comenzar con el cálculo de la entalpía de formación. 

\subsection*{Clase Method}
La clase Method será utilizada para seleccionar el tipo de método que se realizó en \textit{Gaussian}, y así, devolver valores específicos para los átomos existentes en la molécula analizada.

\subsection*{Clases EnthalpyG4 y variantes}
El trabajo de esta clase y sus variantes (EnthalpyG3, EnthalpyG3MP2, EnthalpyCBS-APNO y EnthalpyCBS-QB3) será realizar las operaciones aritméticas para obtener el calor de formación de la molécula, la entalpía de formación a T = 298.15 K por el método de atomización y la entalpía de formación a T = 298.15 K con correcciones en la energía interna. Para concluir, imprimirá los resultados a través de la línea de comandos.

%-----------------------------------------------------------------------------% 
% Lugares y Cronograma
%-----------------------------------------------------------------------------% 
\section*{Lugar y tiempo de realización}
El trabajo de tesis se llevará a cabo en el Laboratorio de Modelado Matemático y Desarollo de Software Científico y en el Laboratorio de Fisicoquímica Orgánica Teórica de la Facultad de Ciencias Químicas de la BUAP, en un periodo de 12 meses.
\vspace{1cm}
\begin{table}[hbp!]
\centering
\footnotesize
\setlength{\tabcolsep}{2.0pt}
\begin{tabular}{||p{0.4\linewidth}|c|c|c|c|c|c|c|c|c|c|c|c||}
\hline
\textbf{Actividad} & \multicolumn{12}{|c||}{2021-2022}\\\hline
& Ago & Sep & Oct & Nov & Dic & Ene & Feb & Mar & Abr & May & Jun & Jul\\\hline
Revisión bibliográfica & X & X & X & X & X & X & X & X & X & X & X &  \\\hline
Curso de Termodinámica Estadística & X & X & X & X & & & & & & & &  \\\hline
Cursos de \texttt{Bash y C++} & X & X & X & X & & & & & & & &  \\\hline
Inspección de código inicial & & & X & X & X & X & & & & & & \\\hline
Redacción de tesis &  &  &  &  & X & X & X & X & & & & \\\hline
Periodo de pruebas y correción de bugs & & & & & & & X & X & & & & \\\hline
Verificación y corrección de tesis & & & & & & & & & X & X & X & \\\hline
Examen &  &  &  &  &  &  &  &  &  &  & X  & \\\hline
\end{tabular}
\end{table}
\newpage
%-----------------------------------------------------------------------------% 
% Bibliografía
%-----------------------------------------------------------------------------% 
\bibliographystyle{./general} 
\bibliography{./listado}
\end{document}
